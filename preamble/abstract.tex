\unnumberedchapter{Abstract} 
\chapter*{Abstract} 
\subsection*{\thesistitle}



\hspace{10pt}Automation today is no longer a luxury but rather a necessity for economies to thrive. Through globalization, it has helped integrate economies around the world. Some workers reap significant benefits from technological progress, while some are displaced and struggle to find new ways of survival. Although the world's economic progress is growing tremendously, this Skill-Biased Technological Change (SBTC) has contributed significantly to wage inequality in developed and developing economies. It raises the question: What level of inequality is acceptable, and what is the effect of this widening gap between the rich and low across the globe has on economic growth and prosperity? This project aims to answer these questions by creating a panel of forty developed and developing economies to assess the relative contributions of Skill-Biased Technological Change (SBTC) and globalization towards income inequality. We implement this assessment by creating a LASSO-based regression analysis where we use the GDP per capita (measure for Income Inequality) as our dependent variable and factors influencing SBTC and Globalization as our independent variables. With our proposed research, we wish to understand if we have been blindfolded by technology advancements and what the current trends could cost to the future of the global economy.