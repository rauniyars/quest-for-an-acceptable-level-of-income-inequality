\chapter{Related Work} 
\label{ch:relatedwork}

\hspace{20pt}The related work section will focus on the various scholarly research done to support the argument that Skill-Biased Technological Changes (SBTC) and Globalization contribute to an increasing wage inequality within developed countries and developing economies. These analyses help us discuss the similar approaches taken by these authors to study the factors affecting income inequality, and the aim is to use their findings to create a basis for this senior project.

\label{sec:relatedwork}
\section{Essence of Income Inequality}

\hspace{20pt}The rising gap between the rich and the poor has fueled unrest around the world. During the most important events in history, the Industrial Revolution, there was a rapid transition to the modern age resulting in a sustained rise in the real income per person in the Western World \cite{nardinelli}. The long formative process of the western capitalist economies rose income inequality during the initial stages of the industrial revolution, primarily in rich countries like Britain and the United States \cite{ryckbosch2016economic}. Low-income countries were not even a part of the game as the process of industrialization was not even uniformly introduced within their economies. This difference in the introduction and adoption has produced inequities among nations and among people on a scale that was never experienced, hence resulting in a spike in economic growth in more prosperous economies while the poor ones are trying to catch up.


The most influential modeling of the relationship between economic growth and inequality was presented first by Simon Kuznets, where he proposed an association between levels of income per capita and inequality in the shape of an inverted U-curve. In the empirical work by Jeffrey and Williamson, the “upswing” in the Western world was associated with two processes: unbalanced technological change (causing growing income differentials between different economic sectors and driving up the skill premium for human capital) and demographic growth \cite{korzeniewicz2005theorizing}. For the “downswing” of inequality in the developed world, globalization was said to have played a significant role. Many countries have followed this up and down trend; however, with time, they are various changes, including government policies, laws, demographics, and most importantly, technology, that prompt us to think well beyond the curve.


Whether it is the agricultural revolution or the industrial revolution, what remains common is innovation. Currently, we lay on the edge of a new revolution- the technological revolution that has fundamentally changed the way we live, work, and connect. On the one hand, technology has become the engine of growth, leading to a more globalized economy where people have become successful, more affluent, and educated. On the other hand, the benefits of rising incomes and aggregate GDP growth rates have not been shared equally across all segments of the population.  The advent of new technologies leads to more job and income opportunities, but the skill-biased nature of the new jobs complements the higher-skilled workers whereas the low-skilled are bound to be replaced. There is no doubt that these changes have adverse effects on the economy.

\section{Evidence of Skill-Biased Technological Changes in Developing Countries}

\hspace{20pt}When economies are introduced to new technologies, there is a change in production methods or modifications in the organization of work. Hence, to have a smooth transition, firms turn towards more skilled labor.

\hspace{20pt}In the research paper titled 'Skill-Biased Technology Transfer,' authors Berman and Machin investigate the skill-bias of technological change in developing countries using a global sample of manufacturing industries \cite{berman2000skill}. The manufacturing industries undergo drastic structural changes as they transition to new forms of technology. In this paper, we turn back to the 1980s and find strong evidence of increased demand for skills in the 1980s in middle-income countries' manufacturing sectors. This analysis links this demand shift due to the skill-upgrading within industries. A cross-country correlation analysis was conducted for 37 countries with data (primarily from the United Nations General Industrial Statistics Database) on employment, wages, and manufacturing industries. The results showed that the demand for skills accelerated in the manufacturing industries of middle-income countries in the 1980s to a rate matching even that of high-income countries, concluding that skill-biased technological change exists, thus supporting the basis of this thesis \cite{berman2000skill}.

In addition to an analysis focusing on multiple countries, it is also helpful to understand the situation between individual economies. Numerous studies focus on individual countries that help create an in-depth analysis of the impact of SBTC changes under various circumstances. India is preparing itself to compete with the leaders of industrialization and technological development. With this economic prosperity, there are also rising concerns about the increasing inequality within this fast-track nation. Kijima Yoko conducted a study to identify the major causes of changes in the wage structure in urban India during the 1980s and 1990 by conducting surveys that include information on an individual's earnings and the labor market characteristics \cite{kijima2006did}. The results showed that the demand for skilled workers grew faster than the supply of skilled labor, which raised the returns to skills, resulting in accelerated inequality in the 1990s. Another study by Esposito and Stehrer explains relative demands for skills in Central and Eastern European countries. It presents the hypothesis that the sector bias of SBTC is essential in defining the rising skill premium in three transition economies: Czech Republic, Hungary, and Poland \cite{esposito2009sector}. Another study investigating the existence of SBTC uses Ethiopia to study the effects of imported technology on manufacturing employment and concludes with similar remarks of biased outcomes \cite{haile2017imported}.

In areas such as the MENA region, political instability is already a factor that worsens income inequality. A study was conducted to determine the effects of technological advancements measured by the demand of patents on skilled and unskilled labor through a dynamic panel of countries within the MENA region. The sample data composed of six countries, including Egypt, Algeria, Iran, Malta, Morocco, Tunisia, for the period 1965-2010. Patents, expenditure on Research and Development, and the value of imported technologies were the key variables used in this analysis. There was a positive effect on skilled labor, while the unskilled labor force experienced a negative effect. Overall, the political and economic instability within the MENA region makes income inequality a huge issue, and these findings show the importance of addressing this pressing issue. Within the MENA region, Tunisia is taking steps towards becoming a technology-consumer country. However, there are many obstacles to implementing technological innovations within the country, like high costs, lack of human capital, poor infrastructure, etc. Such reduced technological potential within most developing economies like Tunisia enforces them to depend on foreign technologies. These organizational changes within the industry have resulted in high wage disparities, thus proving biased due to technological changes \cite{aissaoui2015skill}. Emerging economies are taking initiatives to catch up with developing nations. In a research paper focusing on the big question of whether Skill-Biased Technological Change exists in Vietnam or not, data was collected from the Vietnam Household Living Standard Survey from 2004 to 2014 to measure the relative skill productivity. The research involved developing a regression model for estimating the wages of skilled and unskilled workers. The results concluded an increasing wage gap that is bound to increase wage inequality. An important insight is that Vietnam is an attractive destination for outsourcing due to its cheap labor. The country has been actively investing in high-tech machines and equipment, leading to the inevitable consequence of high-tech capital investment. Such tactics led by developing economies is taken to give them a competitive edge and boost economic growth. This results in the favoring of skilled workers and eventually increasing the wage differentials \cite{nguyen2018skill}.

\section{Skill-Biased Technological Change in Developed Countries}

In most economically advanced countries, the trend towards income-inequality remains the same. Historically speaking, the Kuznets curve speaks for the relationship between economic development and income inequality. It follows a “U-shape” from 1935 to the mid-1950s and then stability until the 1970s. Then, inequality took off and has increased ever since. The rich economies: U.S. and U.K. have followed this trend quite strictly, and the results are obvious. In 2018, the top 20\% of the population earned 52\% of all U.S. income \cite{bureau}. Including the United States, most of the high-income countries are a part of OECD. OECD stands for Organization for Economic Cooperation and Development. It's an association of 37 nations in Europe, the Americas, and the Pacific to promote economic welfare. Following the U-shaped curve trend, income inequality has been on the rise in most of the OECD and in many emerging economies since the 1980s, and skill-biased technological changes give it momentum \cite{goldin2009race}.

The Economic Department of the OECD presents a study that estimates skill-biased technological changes as the key driver of the increasing earning differentials. The research shows a model that learns from the historical trends to construct scenarios for the years leading up to 2060. The model predicts that if the current trend of SBTC observed during the past 25 years prevails, the earnings differentials will, on average, increase by almost 30 percent by 2060. The study further supports the ideology that income inequality patterns differ not only across countries but also within countries and educational attainment and globalization exert upward pressure on income inequality \cite{braconier_ruiz-valenzuela_2014}. Another study explains how skilled refers to the educated and explains the concept using the shift in the supply-demand curve. At any point, the demand for skilled workers is considered to be fixed. Technological changes create an upward shift in the demand for skilled workers, and since the demand is greater than the supply, it leads to an increase in the relative wage of the skilled worker, thus increasing the gap between the high-skilled and the low-skilled. This economic phenomenon is explained through the trends in recent decades within multiple OECD countries, including Germany, France, etcetera \cite{atkinson2007distribution}. While many empirical studies focus on a specific industry (for instance, manufacturing) over individual economies, this debate was widened by cross-country analysis. The study focused on the impact of SBTC and International Trade (IT) on skill-based wage inequality carried across a panel of 25 OECD countries from 1997 to 2006. The sample was further split into two groups: developed countries and developing countries. In the study, the expenditure on Research and Development was used to proxy the effects of SBTC, while Foreign Direct Investment (FDI) played a crucial role in analyzing International Trade (IT). The results suggested that SBTC was the main reason behind inequality in developed countries, while in developing countries, IT dominated that role \cite{almeida2010sbtc}. Openness to trade allows emerging economies to implement technological advancements similar to those within the developed economies, so is it fair to see that soon the developing economies will find themselves in the shoes of the developed economies.

The situation in some of the world's largest economies is not different. In an empirical analysis conducted by Cristiano and Franceso, the rate and direction of technological change in a significant sample of 12 major OECD countries in the years 1970–2003 confirm the strong bias of new technologies \cite{antonelli2010effects}. The study further supports another study's findings that the same country experiences different levels of economic growth at different periods, and these can be identified by the changes in the direction of technological change \cite{david2005tale}. New and convincing evidence has been provided recently about the strong skill-bias of the gales of technological change based upon information and communication technologies introduced in the last decades of the twenty-first century \cite{goldin2009race}. It has been a key driver in increasing the differences between people's earnings. Another model decomposes historical changes in earning differentials and uses it to construct forward-looking scenarios up to 2060. If the common cross-country trend of skill-biased technological change observed during the last 25 years prevails, earning differentials will increase by almost 30\% on average in the OECD (Organisation for Economic Co-operation and Development) \cite{braconier_ruiz-valenzuela_2014}. Skill-biased technological changes can have a dramatic effect on wages and the labor market. An examination of the impact of the SBTC on the demand of productions and non-production (unskilled and skilled) workers of the Japanese manufacturing industries was implemented through cross-sectional regressions. It showed that increased investments in computers significantly impacted the share of the wage-bill held by non-production workers []. The study defined production workers as people who are engaged in production at manufacturing establishments. While on the other hand, the non-production workers included people working at higher positions like supervisors, technical employees, and other office work \cite{sakurai2001biased}. The study predicts that demand for non-production or skilled workers will continue to accelerate in the years to come due to the direction of SBTC.

\section{Combining Skill-Biased Technological Change with Globalization}

\hspace{20pt}The recent accelerations in the advancements of technology have also increased the globalization of production. Technology has revolutionized the global economy and has become a critical competitive strategy \cite{lamba2009role}. Through globalization, companies benefit from free trade, and consumers enjoy low prices. However, it has also created structural changes in the economy. The National Bureau of Economic Research states that trade influences what kind of technologies are more profitable to develop and tends to increase the price of skill-intensive products. It provides firms with incentives to introduce new skill-biased technologies within their manufacturing processes. In other words, trade and globalization induce skill-biased technological change \cite{acemoglu_2003}.

Technology creates a sense of competitiveness that triggers a race to imitation and innovation for both firms and countries. Both firms can feel obligated to catch up with the technological trends in their production processes to ensure they are not lagging. However, they do so at the cost of a larger share of unskilled labor in their workforce. This argument is supported by a study conducted by Thoenig and Verdier, where they show that when globalization triggers an increased threat of technological leapfrogging or imitation, firms tend to respond to that threat by biasing the direction of their innovations towards skilled labor-intensive technologies \cite{thoenig2003theory}. Another finding is that there is a  growing international trade integration between advanced economies and low wage countries. According to standard Heckscher-Ohlin theory, it has shifted labor demand away from unskilled workers in high wage economies and eventually create a bias towards high education workers \cite{leamerheckscher}. To understand the consequences of this integration, they conduct their analysis in two regions, North-North and North-South, and conclude an increase of wage premium in both areas.

Another study focuses on how lower-middle-income countries (LMIC) rely on international technology transfer to upgrade their technological sectors. In this analysis, a Generalized method of moments (GMM) technique is applied to a panel data constituting 0f 20 manufacturing sectors for 23 countries over a decade. The results provide evidence of widening employment differentials. They introduce a new variable to overcome one of the challenges mentioned in their study: the absence of innovation and employment data in low- and middle-income countries. The results indicated a significant increase in skilled workers' demand for imports of industrial machinery, equipment, and ICT (Information and Communication Technology) capital goods. Hence, the findings support this senior thesis's arguments that technological advancement and trade directly or indirectly impact the inequality within workers \cite{conte2011imported}. It is assumed that due to their pace, developed countries are generally abundant in skilled-labor. When they increase their trade with developing countries where unskilled labor is abundant, it raises the skilled workers' wage relative to the unskilled workers. According to the Stolper-Samuelson theorem, the increasing trade should lead to a decline in wage inequality; however, the scenario is different \cite{gorg2002relative}. A study using plant-level data in Mexico by Hanson and Harrison found that although trade barriers were reduced, there was evidence of rising wage inequality [same as above]. Another study studied the impact of technological changes across various firms in Mexico, Columbia, and Taiwan and found that the employer-size-wage effect is higher for skilled workers than for unskilled workers in technology investing firms. Such firms invest in exporting and conducting R&D and provide training to their employees. The argument presented was that technological changes are skill-biased and that larger firms are more technology-intensive \cite{gorg2002relative}.

Lower Middle Income Countries (LMIC) are not the only ones struggling. The results are prominent within developed economies where globalization drives economic growth. Innovation in such economies requires a significant amount of skilled labor, and this is an attractive feature for multinational corporations (MNC) who are willing to invest in these sectors. According to the 2019 World Investment Report, China was ranked the world's second-largest Foreign Direct Investment (FDI) recipient after the United States. In a study that performs regression analysis to firm-level data from the Chinese electronic industry, the findings show that despite the high adoption cost of technology, multinational firms' investments lead to higher productivity of skilled labor in developed economy subsidiaries than in the emerging economies \cite{li2010multinational}. The study concludes by reiterating the point mentioned above that innovation in developed economies favors skilled labor. It is more costly for MNCs of developed economies to invest in innovation techniques of unskill-biased technologies and customize their production facilities within developed economies \cite{li2010multinational}.


\section{Statistical Models within Machine Learning: LASSO Regression}

\hspace{20pt}Most works in Economics focus on the application of traditional statistical methods. Today, innovative techniques within Artificial Intelligence (AI) like Machine Learning (ML) have transformed the way people live. Netflix's show recommendations, Instagram's likable posts, Apple smartwatch that tracks health, and Siri's voice-recognition system that responds to user's commands are a few examples of how ML has made our lives easier. Nevertheless, many economists have not embraced this new and innovative technology. Unlike the standard econometric models, the predictive algorithms that make Siri smarter over time cannot answer questions about correlation and causation. With the bulk amount of data that we have today, we must turn our heads to the innovative methods in Machine Learning that provide accurate trends. When it comes to making predictions, the existing techniques tend to "over-fit" the data sample and make flawed generalizations of new and unseen data. Hence, this focus on the predictions' accuracy is where machine learning algorithms play an essential role. Materials scientists seek to use these learning algorithms to easily and efficiently apply to their data to obtain quantitative property prediction mode \cite{mannodi2016critical}.

Machine Learning models can minimize the forecasting error by trading off bias and variance and handling vast amounts of data focusing on prediction problems. In addition to over-fitting, the traditional methods also create an overestimation of how well the model performs using the included variables to explain the observed variability (optimism bias). Such problems can be addressed by various penalized and regularization regression models, and some of these models are Ridge and LASSO (Least Absolute Shrinkage and Selection Operator) \cite{ranstam2018lasso}. In this senior thesis, I plan to use LASSO regression since it leads to improved model accuracy. Tibshirani introduced this regularization method to subset the variables from the original list based on the extent that they impact the model \cite{tibshirani1996regression}. The model has been applied in a wide range of disciplines.

In Bioinformatics, recent technological advancements have been highly captivated by the computational techniques and algorithms. In this study, they propose the development of a fused lasso logistic regression (FLLR) to differentiate the patients of early Alzheimer's disease (AD) from normal controls based on the corpus callosum (CC) thickness profiles. The callosal thicknesses are spatially correlated because the thickness at one point is correlated to the thicknesses of its neighboring point. The study focuses on using fused lasso regression to select continuous regions rather than individual thickness points to differentiate the groups. The accuracy of the classifier was estimated to be 84 percent based on five-fold cross-validation \cite{lee2014application}. Recent scientific and technological advancements show the emergence of MicroRNAs (miRNA) as a significant class of regulatory molecules involved in a broad range of biological processes and diseases ranging from Alzheimer's to Diabetes. With the increasing amount of miRNA and gene expression data, a study conducted to accurately identify miRNA-mRNA pairs proposed the LASSO regression model. It concluded it to be a robust tool as it deals well with sensitivity and specificity when used for the diagnoses and the treatment of complex diseases \cite{lu2011lasso}. The LASSO method is also widely used within high-dimensional data. When combined with penalized logistic regression, LASSO was used in high-dimensional cancer classification. It led to a relatively efficient and feasible classification by selecting fewer genes with a high area under the curve and a low misclassification rate \cite{algamal2015penalized}.

A study conducted on predicting corporate bankruptcy implemented Lasso and Ridge regression to design a model for an extensive data set of 2032 non-bankrupt firms and 401 bankrupt firms belonging to the hospitality industry over the period 2010-2012. The algorithms were used because they deal better with multicollinearity and display the ideal properties to minimize the numerical instability due to overfitting \cite{pereira2016logistic}. The model used plenty of variables, and it became essential to identify the most critical variables in a data set. Using Lasso regression allows us to shrink the parameter estimates close to zero or even zero, excluding some of the model variables. Such implementations improve the model's prediction accuracy, leading to Lasso's outperformance against existing statistical methods \cite{wang2018predicting}. Numerous studies widely use the logistic regression method in economics. However, Hosmer and Lemeshow report that although logistic regression copes with many restrictive assumptions of ordinary least squares regression that include linearity, normality, and heteroscedasticity, it cannot handle multicollinearity \cite{hosmer2013applied}. With this concept, Lasso was also used to create a model that ensures economic sustainability by predicting franchisor success or failure based on numerous financial and contractual variables \cite{calderon2017economic}. The model correctly predicted 82.47\% of franchisor survival in the training sample and 81.82\% in the test sample, supporting the model's accuracy. In addition to this, the LASSO regression has also been an attention-grabber in the stock-market alongside investors. To compare its efficiency against the traditional linear regression techniques within the stock market, a study was carried out that was composed of daily closing stock price in Indonesian Stock Market over the years 2000 to 2014. The study focused on analyzing the effect of the stock market of G7 Nations and the Association of Southeast Asian Nations (ASEAN) on the Indonesian stock market. There was high correlation and multicollinearity within the variables; hence the solution is geared towards shrinking the parameter coefficients, aka LASSO regression. LASSO beat the linear regression techniques using the least-squares method and became the best model in the Indonesian Stock Market \cite{setiawan2018lasso}.

LASSO has also made some great appearances in the field of social science. In the heterogeneous world of social sciences, the common method uses a multiplicative interaction term between the treatment and a hypothesized effect modifier to evaluate theoretical arguments. These standard methods use a "base-line" regression model to identify the relationship between interest variables and the hypothesized moderator. However, biased estimates have been critical problems, eventually leading to overfitting and unstable estimations. Data-driven approaches within Machine Learning can be used to address the direct and indirect regularization bias as conducted in a study that uses a post-double selection approach that uses several Lasso estimators to select the interactions to include in the final model \cite{blackwell2020reducing}. According to the evidence from the simulation, the LASSO-based model showed better performance.  In another study conducted on US midterm elections, it was found that due to low survey response rates,  there has been greater use of non-probability samples, which may lead to selection bias. The study estimated the voting preference for 19 elections in the US 2014 elections using non-probability surveys from SurveyMonkey. The model was adjusted to estimated control totals using a model-assisted calibration combined with adaptive LASSO regression (also called estimated controlled LASSO, ECLASSO). This methodology was shown to be powerful compared to the traditional calibration methods and has been proposed to be applied across various social science and health research disciplines as they struggle to receive probability samples \cite{chen2019calibrating}.

What makes the LASSO technique highly attractive is that it improves predictions by shrinking the regression coefficients. This feature has helped accurately forecast the variability in solar irradiance that reaches the ground by moving clouds. The study was conducted in Hawaii to a second irradiance time series data, and LASSO was used because its bias-variance trade-off leads to better predictions. LASSO  considers the sum of l1-norms of the regression coefficients as a penalty, which shrinks the regression coefficient. This feature suits the application in this study where one can control the down-wind stations' influence on the up-wind stations in a highly correlated data set \cite{yang2015very}. Few more studies provide evidence of LASSO's prediction techniques; one of the models is used in remote sensing. The model carries out a  biomass estimation using satellite data in a region with extremely low vegetation cover. Multiple empirical models are applied to test the ability to deal with a large dataset, and LASSO was shown to outperform all the models. Working with an extensive data set of satellite-based predictors shows how efficient the Lasso model can be. Another climatological research uses LASSO regression to identify systemic biases in decadal precipitation predictions from a high-resolution regional climate model (CCLM) for Europe. The algorithm was applied to observe precipitation and numerous predictors related to precipitation derived from a training simulation. This trained Lasso regression model was then transferred to a virtual forecast simulation for testing. LASSO outperformed the local predictors, and hence his bias adjustment has been said to contribute to improving the seasonal cycle of precipitation prediction \cite{li2020comparing}.


The feature of variable selection is crucial in some sensitive areas such as medicine. For instance, in epidemiology, it has been used to detect collective exposure from a pool of candidate variables diagnosed with Hepatitis B. Implementing the LASSO regression technique in this study helped detect the crucial factors that resulted in the infection, and intensive simulations were used to compare this method with previously researched methods\cite{guo2015improved}. LASSO can also be used to implement models with high-dimensional data and hence reduce the amounts of computation. To identify the risk of disease recurrence and the benefit of adjuvant chemotherapy for patients who have had surgery for stage II colon cancer, Zhang built a model using the LASSO regression from the high-dimensional microarray data \cite{zhang2013prognostic}. Another model utilized the LASSO regression for selecting variables from multiple factors that influence energy consumption in residential buildings \cite{satre2019investigating}. Another crucial application aligns with climate change, which is currently one of the most critical global concerns. Household Carbon Emissions (HCEs) have contributed significantly to the rise in CO2 levels in the atmosphere. There are many factors of HCEs; however, a clear understanding of the driving factors achieved through LASSO regression would be critical for policymakers to rank the factors according to their importance and make required modifications to their policies \cite{shi2020prioritizing}. With the amount of data and problems increasing globally, it is worth noting that setting our aims high and resolving everything might not be a great tactic. However, what could help is identifying the pressing issues and finding solutions to them, and Lasso regression can be useful for such implementations.




