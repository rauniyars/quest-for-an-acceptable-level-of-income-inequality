\chapter{Introduction}
\label{ch:intro}

\emph{“It's difficult to cross from one economic class to another. You'll drown in champagne.”}

\hspace{20pt}In today's time, technology has embedded in our day-to-day activities and businesses. As it changes the way we access information, consumer products, travel, and communicate, it helps us achieve the so-called impossible and makes our lives convenient. It has also changed the model of work. Both developed and developing economies have made transitions from agricultural to industrial and from service to digitalization. The essence of all of these revolutions that humankind has seen lies within the realms of automation. Emerging technologies, including artificial intelligence, machine learning, and advanced robotics, can automate many tasks currently performed by workers, leading us to a quiet revolution, the Automation Revolution \cite{tobenkin_2019}.

There is much debate about where workplace automation will lead the economy, but observers tend to agree on one thing: The trend is only gaining momentum \cite{uzialko_2019}. Automation in infrastructure and development is no longer a luxury but rather a necessity for humankind to thrive. As firms adopt new methods of production, markets expand, and societies evolve. Economies have continued to grow, and while many will benefit, that growth will not be costless. Some arguments say automation will take away jobs, but the advancements in automation do not eliminate jobs but instead create a "shift" reducing particular job functions at which humans are inefficient or inconsistent or are exposed to risk \cite{tobenkin_2019}. Workers in some sectors benefit handsomely from technological progress, whereas those in others are displaced and have to retool to survive \cite{sumagaysay_2019}. One of the dramatic changes in society because of such technological progress is Skill-Biased Technological Change (SBTC).

\section{The Human Effect of Income Inequality}

At the beginning of the 1970s, economic growth slowed, and the income gap widened. Today income inequality is known as the "defining challenges of our time" \cite{nationalarchives}. Humankind today may be proud of all the technological advancements made, but besides all this progress, we might be failing to address a pressing issue that adversely affects human rights. Technology claims to make the world more connected than ever, but I find it amusing that we tend to grow further and further apart on the spectrum of wage and inequality. The world might be reaching its peak of development, but some communities are left in the same spot they were in years ago. When we say progress, we have been blindfolded by the top 10 percent of the world's speedy progress while the rest of the world takes small steps. Coming from an underdeveloped country, Nepal, to one of the world's biggest economies, the United States of America, I have seen, felt, and experienced the drastic difference that lies within these economies. The rich in the Nepalese community is nowhere close to the wealthy in the U.S. The situation is worse for the poor. As Nepal takes baby steps towards technological development, the U.S. boasts of its immense progress and so is the case in other economies and hence it is important to understand the value of global prosperity. There is no doubt that technology has transformed the ways we live, but why the question is, why does it have to be at the stake of human labor? Humankind has already established enough differences based on race, religion, and social norms, and one could say that the essence of these categories is somewhat based on income inequality. It is high time that we break these barriers and unite to create a better world that values peace and prosperity.


Inequality is growing for more than 70 percent of the global population, exacerbating divisions' risks and hampering economic and social development \cite{unitednations}. The World Social Report 2020, published by the United Nations, shows that the richest one percent of the population are the big winners in the changing global economy. It is a mind-boggling insight that the bottom 40 percent of the world have earned less than a quarter of the income in all countries surveyed \cite{unitednations}. The wide disparity in wages spreads to health and education areas and is one of the primary reasons people still lie within poverty walls. Technological advancements that complement the skilled workforce make it even worse. It is more likely to be profitable to people with jobs that require high social and analytical skills, and these are jobs that have high earnings already. Workers in mid-to-low-skill roles who rely on physical labor or analytical skills vulnerable to automation are at higher risk of losing their jobs or facing pressure on wages. If recent history is a guide, those who lose their jobs may face lower incomes throughout their career after being reabsorbed into the workforce, and some may choose to drop out entirely \cite{harris_kimson_schwedel_2019}. The results of job loss and wage suppression will ultimately lead to income inequality and disrupt the economy.


If the current trend continues, one could wonder what the future of the world's economy will look like. Within communities and beyond, what does it look like if the trend gains momentum. It is high time that we treat this issue at the national and international level. Whether it be within countries or beyond, leaders of all communities need to implement policies that can help bridge the gap between the poor and the rich. To tackle within country inequalities, it is required to increase fiscal policy space at the national level to enact the country-specific mix of policies needed to lift all boats and, in particular, to increase the income of those at the bottom. From an international economic system perspective, these imbalances can be addressed by requiring global financial, investment, trade, monetary, and fiscal reforms to reduce volatility \cite{carpentier_wright_passos}. There should be a just system that treats everyone equally, and this should be done before income inequality becomes the sole reason for political and social inability. And I aim to use my knowledge within the areas of Economics and Computer Science to spark light on the intensity of the issue through this senior project. Besides, work is not just a means of earning; for some people it is a purpose in life. We, humans, have the tool of intelligence that makes us a dominant species of the planet. But how inhumane is it for us to build a progressive economy on one end of the spectrum while on the other end,
we are failing to address the silent voices of the people whose lives are being affected. Are we failing to ignore that technology might be robbing humanity? It is important to demand a sustainable balance and that starts by understanding the consequences that have led to people losing their jobs and the increasing wage gap between the rich and the poor.

\section{Skill-Biased Technological Change and Globalization within Economies}
The recent consensus is that skill-biased technological change favors more skilled workers, replaces tasks previously performed by the unskilled, and exacerbates inequality \cite{acemoglu2002technical}. Countries worldwide have undergone structural changes and although these changes contribute to rapid growth, leaning towards highly skilled workers puts the low-skilled workers at risk. The early life of automation begins, then, with the industrial revolution and industrial machinery between 1790 and 1840 \cite{thinkautomation_2020}. Even the people were in fear of having their jobs taken away however, things ended well as there was a rise in the productivity of lower-skilled workers, and when technology evolved during the 20th century, the rise of the computers benefited the higher-skilled workers. Many economists say that there should not be any fear of technology. They point to how past major transformations in work tasks and labor markets – specifically the Industrial Revolution during the 18th and 19th centuries – did not lead to major social upheaval or widespread suffering. These economists say that when technology destroys jobs, people find other jobs
\cite{vardi_2019}. Switching to a new job might be easier for those who are sophisticated enough to learn the skills; the high-income bearers. The lower-income bearers barely make enough to support their basic needs and expecting them to be on track with their skill set is amusing to me.

Another reason for the increase in demand for high-skilled labor is trade \cite{kang2002technological}. The growing interdependence between economies resulting from globalization has created a worldwide market for companies and consumers to benefit. However, the general complaint about globalization is that it has made the rich become richer while making the poor get poorer \cite{collins_2015}. With all these changes that foster the world's economic progress and contribute to the wage gap between the rich and the poor at the same time, the question becomes, "What level of inequality is acceptable, and when does it start doing harm?" It is not acceptable to turn a blind eye to these changes, and this project aims to spark light on this issue by doing a comparative analysis between the developing countries and the developed countries and aims to understand the future of the world's economy.  



\hspace{20pt}The economic growth of a country is influenced by advanced technological changes and the rise in free trade. However, "Is the rise of the machine, the fall of the worker?." As businesses lean towards automation, the machines become more sophisticated, and such a change requires firms to hire high-skilled workers. Since this biased direction of technological change reduces the working capital's stock by substituting capital and unskilled labor with skilled workers, is this fair progress? Countries are using technologies to produce more in less time to match their increasing demands domestically and internationally. Advanced economies like the U.S and China who have a competitive advantage among many goods, have an increasing demand globally. As nations follow this incentive and produce more goods and services on a larger scale, they have a better chance of decreasing their costs and achieving economies of scale. Investments follow this trend of reduced costs that automate tasks as the goal is to "produce more in less time." Developing nations follow the same steps and integrate technological advancements to meet their local demands and open a gateway to foreign trade.

On the bright side, SBTC and globalization bring economic prosperity to all the countries; however, the more countries experience such changes, the greater is the incentive and the potential to eliminate lower-level jobs and decrease the relative wages of the less-skilled workers. Another reason why this happens is that countries aim to achieve economies of scale by reducing their input costs. As a result, they have accumulated enough to invest this sum into a new resource; hence, the cycle continues. Critics argue that this is more of a “shift” in jobs rather than elimination. However, we fail to address that it is easier for the high-skilled workers to adjust to this change as they have access to resources to help them “catch up.” On the other hand, the low-skilled do not have the privilege to make a swift switch mostly because they do not have the resources like sufficient income to invest in their education, primarily why they had to work at a low-skilled job in the first place.


\section{Theoretical Framework}

\hspace{25pt} According to research by McKinsey, it is estimated that between
400 million and 800 million individuals globally could be displaced by automation and need to find new work \cite{McKinsey}. However, finding a new job is easier said than done. Research suggests that jobs that are threatened by technological changes are highly concentrated among lower-paid and lower-skilled workers who usually have limited access to education. This shows how automation will continue to create pressure on workers at the lower end of the spectrum as they struggle to learn skills that could help them adjust to technological changes. So people with lower wages not only end up losing their jobs but are challenged to find a new work that goes beyond the scope of their skills and they do not have the resources to upgrade the skills. We are told to welcome automation rather than fear it but the ‘we’ in
this is merely focused on the wealthy folks who are fortunate to make choices and chances to change their life swiftly. However, the rest dread as there is nothing barely anything being done to make things better.

Developed economies are insanely rich, and they could invest in their human capital to make their workers prepared for the transitions. Developing economies, who are still beginners in this game, do not seem to have the resources, and hence their workers are at a loss. Hence we could reiterate that "the rich keep getting richer, and the poor get poorer." Now the question is, what does this trend look like in the long-term for the world's economy? The purpose of this senior thesis is (1) to create an analytical model to understand what are the most important variables influencing SBTC and Globalization that drive income inequality (2) to derive insights from the existing patterns and use them to create a predictive model for the future, and (3) to finally use the insights that could be used to create better future models that emphasize how the current rate of income inequality is unacceptable, and it is high time that we shake the people in power to take actions against the injustice going on. Most importantly the goal is to use the advancements in technology (such as Machine Learning) to solve a real-world problem that has been influenced by technology itself. Technological advancements are there to support the progress of humanity; hence the rise of the machine cannot be the fall of the worker \cite{yglesias_2015}. 
